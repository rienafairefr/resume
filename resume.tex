\RequirePackage{ifthen}
\documentclass[9pt]{developercv} % Default font size, values from 8-12pt are recommended
\usepackage{lmodern}
\usepackage{graphicx}
%----------------------------------------------------------------------------------------

\begin{document}

\newif\ifen
\newif\iffr

\newcommand{\en}[1]{\ifen#1\fi}
\newcommand{\fr}[1]{\iffr#1\fi}
\frtrue

%----------------------------------------------------------------------------------------
%	TITLE AND CONTACT INFORMATION
%----------------------------------------------------------------------------------------

\begin{minipage}[t]{0.45\textwidth} % 45% of the page width for name
	\vspace{-\baselineskip} % Required for vertically aligning minipages
	
	% If your name is very short, use just one of the lines below
	% If your name is very long, reduce the font size or make the minipage wider and reduce the others proportionately
	\colorbox{black}{{\huge\textcolor{white}{\textbf{\MakeUppercase{Matthieu}}}}} % First name
	
	\colorbox{black}{{\huge\textcolor{white}{\textbf{\MakeUppercase{Berthomé}}}}} % Last name
	
	\vspace{6pt}
	
	{\huge \en{Software \& things maker} \fr{Maker, things \& software}}% Career or current job title
\end{minipage}
\begin{minipage}[t]{0.275\textwidth} % 27.5% of the page width for the first row of icons
	\vspace{-\baselineskip} % Required for vertically aligning minipages
	
	% The first parameter is the FontAwesome icon name, the second is the box size and the third is the text
	% Other icons can be found by referring to fontawesome.pdf (supplied with the template) and using the word after \fa in the command for the icon you want
	\icon{MapMarker}{12}{Lille\en{, France}}\\
	\input{infos.tex}
\end{minipage}
\begin{minipage}[t]{0.275\textwidth} % 27.5% of the page width for the second row of icons
	\vspace{-\baselineskip} % Required for vertically aligning minipages
	
	% The first parameter is the FontAwesome icon name, the second is the box size and the third is the text
	% Other icons can be found by referring to fontawesome.pdf (supplied with the template) and using the word after \fa in the command for the icon you want
	\icon{Github}{12}{\href{https://github.com/rienafairefr}{@rienafairefr}}\\
	\icon{Twitter}{12}{\href{https://twitter.com/@rienafairefr}{@rienafairefr}}\\
	\icon{Linkedin}{12}{\href{https://www.linkedin.com/in/matthieu-berthomé}{@matthieu-berthomé}}
\end{minipage}

\vspace{0.5cm}

%----------------------------------------------------------------------------------------
%	INTRODUCTION, SKILLS AND TECHNOLOGIES
%----------------------------------------------------------------------------------------

\cvsect{\en{Who Am I?}\fr{Qui suis-je?}}

\begin{minipage}[t]{0.4\textwidth} % 40% of the page width for the introduction text
	\vspace{-\baselineskip} % Required for vertically aligning minipages
	
	\en{Multi-faceted Engineer, from the electrons to the Cloud\\ \faHeart\ building projects, and helping others to build their projects}%
	\fr{Ingénieur multi-facettes, des électrons au Cloud\\ \faHeart\ construire un projet, et aider les autres dans la construction de leur projet}%
\end{minipage}
%\hfill % Whitespace between
\begin{minipage}[t]{0.6\textwidth}
	\vspace{-\baselineskip} % Required for vertically aligning minipages
	\begin{barchart}{5.5}
		\baritem{Python}{100}
		\baritem{Java}{80}
		\baritem{C\#}{60}
		\baritem{\en{Electronics}\fr{Electronique}}{60}
		\baritem{Docker \& devops}{60}
		\baritem{Javascript}{25}
	\end{barchart}
\end{minipage}


%\begin{center}
%	\bubbles{6/git, 4/Kicad, 3/Inkscape}
%\end{center}

%----------------------------------------------------------------------------------------
%	EXPERIENCE
%----------------------------------------------------------------------------------------

\newcommand{\june}{%
  \en{June}%
  \fr{Juin}%
}
\newcommand{\april}{%
  \en{April}%
  \fr{Avril}%
}
\newcommand{\logo}[1]{%
	\includegraphics[height=2em]{#1}%
}

\cvsect{\en{Experience}\fr{Expérience}}

\begin{entrylist}
	\entry
		{\june{} 2017 -- }
		{\en{Full-stack developer}\fr{Ingénieur full-stack}}
		{INRIA, \en{Lille campus}\fr{Villeneuve d'Ascq} \href{https://www.inria.fr/centre/lille}{\logo{INRIA.eps}}}
		{
			\en{Sysadmin, backend, frontend \& Embedded code}
			\fr{Développement serveur, frontend \& embarqué} \href{https://www.iot-lab.info}{\logo{fit-iotlab.png}}
		}
	\entry
		{\april{} 2017 -- \\\footnotesize{freelance}}
		{\en{Electronics \& Laser cutting instructor}\fr{Formateur électronique et découpe laser}}
		{Techshop Lille \href{https://www.techshoplm.fr/le-lieu-lille}{\logo{Techshop.png}}}
		{\en{Training on the electronics \& laser cutting capabilities}
		 \fr{Formations aux capacités électronique et découpe laser}}
	\entry
		{2014 -- 2017}
		{\en{Microelectronics Laser cutting research engineer}\fr{Ingénieur de recherche découpe laser pour la microélectronique}}
		{IEMN \en{Lille campus}\fr{Villeneuve d'Ascq} \href{https://www.iemn.fr/la-recherche/les-groupes/microelecsi}{\logo{CNRS.jpg}}}
		{\en{Maintenance, execution of cutting jobs, development of interface software}\fr{Maintenance, exécution de jobs de découpe, développement d'interface logicielle}\\ C\#, Visual Basic, WPF, Real-time software}
	\entry
		{2010 -- 2013}
		{\en{Research Assistant}\fr{Assistant-doctorant}}
		{CEA-LETI Grenoble \href{www.leti-cea.fr/}{\logo{CEA-LETI.png}} EPFL Lausanne  \href{https://nanolab.epfl.ch/}{\logo{EPFL.jpg}} \\
		\footnotesize{\en{co-supervision}\fr{co-tutelle}}}
		{\en{Research in micro-nano electronics }\fr{Recherche en micro-nano électronique}}
\end{entrylist}

%----------------------------------------------------------------------------------------
%	EDUCATION
%----------------------------------------------------------------------------------------

\cvsect{\en{education}\fr{éducation}}

\begin{entrylist}
	\entry
		{2010 -- 2013}
		{\en{EDMI Doctoral courses}\fr{Ecole doctorale EDMI}}
		{EPFL Lausanne \href{https://www.epfl.ch/education/phd/programs/edmi-microsystems-and-microelectronics/}{\logo{EPFL.jpg}}}
		{%
			\en{Microsystems, Microelectronics}%
			\fr{Micro-systèmes, Micro-électronique}%
		}%
	\entry
		{2009}
		{%
			\en{Physics \& Engineering diploma}%
			\fr{Diplôme d'ingénieur physicien}%
		}%
		{Grenoble INP Phelma \href{http://phelma.grenoble-inp.fr/}{\logo{GrenobleINPPhelma.png}}}
		{%
			\en{Specialisation in microelectronics}%
			\fr{Spécialisation en nano-électronique}%
		}%
	\entry
		{2008 -- 2009}
		{\en{Full-year exchange}\fr{un an en échange}}
		{UCSD San Diego \href{https://ucsd.edu/}{\logo{UCSD.png}}}
		{%
			\en{Full-time coursework\\6 months project on a plasmonic metamaterial}%
			\fr{Cours à temps plein\\projet de fin d'études sur un métamatériau plasmonique}%
		}%
\end{entrylist}

\cvsect{\en{Open-source}\fr{open-source}}

\begin{entrylist}
	\tableentry
		{\href{https://github.com/rienafairefr/pyynab}{pyYNAB}}
		{
			\en{Reverse-engineered library for the API of www.youneedabudget.com, a personal finance web-app}%
			\fr{Reverse-ingéniérie de l'API de www.youneedabudget.com, un site de budget personnel}%
		}
	\tableentry
		{\href{https://github.com/rienafairefr/pyrebrickable}{pyrebrickable}}
		{%
			\en{Library, Data access \& CLI for the API of www.rebrickable.com, a LEGO bricks web-app}%
			\fr{Librairie, base de donnée et CLI pour www.rebrickable.com, une web-app dédiée aux briques LEGO}%
		}%
	\tableentry
		{%
			\en{Various contributions}%
			\fr{Contributions\\ diverses}
		}%
		{
			\en{FlatCAM, a PCB milling application\\OpenAPI-generator, a code generator from OpenAPI specifications}%
			\fr{FlatCAM, logiciel de gravure PCB\\OpenAPI-generator, générateur de code depuis des spécifications OpenAPI}%
		}%
\end{entrylist}

%----------------------------------------------------------------------------------------
%	ADDITIONAL INFORMATION
%----------------------------------------------------------------------------------------

\begin{minipage}[t]{0.4\textwidth}
	\vspace{-\baselineskip} % Required for vertically aligning minipages

	\cvsect{\en{Languages}\fr{Langages}}
	
	\textbf{\en{French}\fr{Français} - \en{native}\fr{langue maternelle}}\\
	\textbf{\en{English}\fr{Anglais} - \en{fluent}\fr{maîtrisé}}

\end{minipage}
\hfill
\begin{minipage}[t]{0.4\textwidth}
	\vspace{-\baselineskip} % Required for vertically aligning minipages
	
	\cvsect{\en{Hobbies}\fr{Hobbies}}

	\fr{Participations à la coupe de France de robotique}\en{Participations to national robotics tournaments}\\
	\fr{Découpe laser}\en{Laser cutting}
	

	
\end{minipage}
\hfill

%----------------------------------------------------------------------------------------

\end{document}
